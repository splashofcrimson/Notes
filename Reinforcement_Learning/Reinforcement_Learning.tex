\documentclass[a4paper]{article}

\def\nterm {Fall}
\def\nyear {2019}
\def\nlecturer {Silver}
\def\ncourse {Reinforcement Learning}

\input{header}

\begin{document}
\maketitle

\tableofcontents

\section{Introduction}
\subsection{Definitions}

\begin{defi} Reinforcement Learning\\

(Policy) Defines the behavior of a learning agent based on environmental stimuli\\

(Reward Signal) The end goal of any reinforcement learning problem\\

(Reward) Feedback to the learning agent, this quantity needs to be maximized\\

(Value Function) Long-run quantification of good action by the policy\\

(Value) Total amount of expected reward that an agent can accumulate throughout the future\\

(Model) Mimics the environment that a learning agent will engage with. RL can be either \textit{model-based} or \textit{model-free}\\

\end{defi}

\subsection{History}
A brief history of reinforcement learning across three separate threads of research.\\

\textbf{Optimal Control:}\\

Late 1950s: The term \textit{Optimal Control} was coined to describe the problem of minimizing/maximizing a dynamical system's performance over time.\\

Bellman 1957a: Dynamic Programming, suffers from the curse of dimensionality.\\

Bellman 1957b: Markov Decision Processes (MDPs).\\

Ronald Howard 1960: Policy Iteration Method for MDPs.\\

Werbos 1987: More interrelation between learning and dynamic-programming.\\

Chris Watkins 1989: Reinforcement Learning with MDP formalism.\\

Bertsekas and Tsitsklis 1996: The term \textit{Neurodynamic Programming} is coined to combine dynamic programming and ANNs.\\

\textbf{Trial-and-Error Learning:}\\

Bain 1850s and Morgan 1894: Basis for learning by 'groping and experiment'.\\

Edward Thorndike 1911: Law of Effect describes the effect of reinforcing events on action selection.\\

Pavlov 1927: Reinforcement is the strengthening of a pattern as per stimulus (reinforcer) with a response.\\

Thomas Ross 1933: Machine that found its way through a simple maze and remembered the path through switch settings.\\

Turing 1948: 'Pleasure-pain system' to impact system configurations.\\

Shannon 1952: Maze-running mouse Theseus uses trial-and-error to find its way through a maze, although the maze itself also remembers the directions of the mouse.\\

Minsky 1954: Analog machine built of SNARCs (Stochastic Neural-Analog Reinforcement Calculators) that represent synaptic connections of the brain.\\

Farley and Clark 1954: Digital simulation of trial-and-error neural network.\\

\textbf{Interjection} --- Confusion over the difference between reinforcement learning and supervised learning caused a drop in research in the 1960s and 1970s.\\

Minsky 1961: Several issues relevant to trial-and-error learning, such as prediction, expectation, and the \textit{basic credit-assignment problem}.\\

Michie 1961: Trial-and-error system to learn tic-tac-toe called MENACE.\\

Michie and Chambers 1968: Improved with GLEE, pole-balancing teacher.\\

Andreae 1969: STeLLA learned by trial-and-error in interaction with an environment.\\

Widrow, Gupta, and Maitra 1973: Least-Mean-Square (LMS) modifications allow it to learn from success/failure signals as opposed to training examples. Called \textit{selective bootstrap adaptation}, learning from a \textit{critic}.\\

Tsetlin 1973: Origins of Learning Automata published posthumously.\\

Cross 1973: Application of Bush and Mostellar learning theory to classical economic models.\\

Tzanakou 1974: Alopex algorithm stochastically detects correlations between actions and reinforcement.\\

John Holland 1975: General theory of adaptive systems based on selection principles.\\

John Holland 1976: Introduced \textit{classifier systems} including association and value functions. Key takeaway: \textit{Genetic Algorithm}\\

Harry Klopf 1981a: Distinction of reinforcement and supervised learning.\\

\textbf{Temporal-Difference Learning:}\\

Minsky 1954: \textit{secondary reinforcers} are stimuli paired with a primary reinforcer like pain - found significance for integration with artificial intelligence.\\

Arthur Samuel 1959: Checkers-playing program implemented using temporal-difference ideas.\\

Klopf 1972: \textit{Generalized Reinforcement} where each component views all inputs as rewards/punishments. This idea was linked with trial-and-error learning.\\

Witten 1977: Combined all three threads of learning.\\

Barto, Sutton, Anderson 1984: \textit{actor-critic architecture} applied to the pole-balancing problem.\\

Holland 1986: Incomporated temporal-difference into bucket-brigade algorithm.\\

Sutton 1988: Separated temporal-difference from control and introduced the $TD(\lambda)$ algorithm.\\

Chris Watkins 1989: Development of Q-Learning.\\

Gerry Tesuaro 1992: TD-Gammon plays Backgammon well.\\

\section{Multi-armed Bandits}
\end{document}