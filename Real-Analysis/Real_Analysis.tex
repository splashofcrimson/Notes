\documentclass[a4paper]{article}

\def\nterm {Fall}
\def\nyear {2019}
\def\nlecturer {W. Rudin}
\def\ncourse {Honors Analysis I}

\input{header}

\begin{document}
\maketitle

\tableofcontents

\section{The real number system}
\subsection{Exercises}
\begin{ex}[1]
If $r$ is a rational $(r \neq 0 )$ and $x$ is irrational, prove that $r + x$ and $rx$ are irrational.

\begin{proof}
If $r$ is a rational, then $r = \frac{p}{q}$, and the claim can be rewritten as $\frac{p}{q} + x$ is irrational. Lets assume by contradiction that this quantity is rational. Then $\frac{p}{q} + x = \frac{a}{b}$. It follows that $$x = \frac{a}{b} - \frac{p}{q} = \frac{aq - bp}{bq}$$ However, this is a contradiction, since $x$ is irrational, which means that $r + x$ must be irrational as well. Using a similar argument, lets assume by contradiction that $\frac{p}{q}x$ is rational. Then $\frac{p}{q}x = \frac{a}{b}$. It follows that $$x = \frac{aq}{bp}$$ However, this is a contradiction, since $x$ is irrational, which means that $rx$ must be irrational as well.
\end{proof}
\end{ex}

\begin{ex}[2]
Prove that there is no rational number whose square is 12.

\begin{proof}
$\sqrt{12}$ is equivalent to $2\sqrt{3}$. By exercise $1$, a rational times an irrational is also irrational. If $\sqrt{3}$ is irrational (since $2 \in \mathbb{N}$, $2 \in \mathbb{Q}$, then it is true that $2\sqrt{3}$. Lets assume by contradiction that $\sqrt{3}$ is rational. Then $$(\frac{p}{q})^2 = 3$$ It can also be assumed that p and q are both relatively prime, since if they are not, an equivalent $\frac{p_1}{q_1}$ exists that's equivalent to $\frac{p}{q}$, but such that $p_1$ and $q_1$ are relatively prime. Now, this can be rewritten as $$p^2 = 3q^2$$ Since $p^2$ is divisible by $3$, and $3$ is prime, $p$ is also divisible by $3$. This means there is some $a$ such that $3a = p$. By some algebra: $$3p^2 = q^2$$ However, this is a contradiction, since both $q$ and $p$ are divisible by 3, although we originally claimed the two were relatively prime. Therefore, there is no rational number whose square is $12$.
\end{proof}
\end{ex}

\begin{ex}[3]
Prove the following proposition given that $x, y, z \in F$, where $F$ is a field.
\begin{enumerate}
    \item[(a)] If $x \neq 0$ and $xy = xz$ then $y = z$
    \item[(b)] If $x \neq 0$ and $xy = x$ then $y = 1$
    \item[(c)] If $x \neq 0$ and $xy = 1$ then $y = \frac{1}{x}$
    \item[(d)] If $x \neq 0$ then $\frac{1}{\frac{1}{x}} = x$
\end{enumerate}
\begin{proof} Parts:
  \begin{enumerate}
      \item[(a)] Applying axiom (M5) $$y = \frac{1}{x}xy = \frac{1}{x}xz = z$$
      \item[(b)] Applying axiom (M5)
      $$y = \frac{1}{x}xy = \frac{1}{x}x = 1$$
      \item[(c)] Applying axiom (M5)
      $$y = \frac{1}{x}xy = \frac{1}{x}(1) = \frac{1}{x}$$
      \item[(d)] Applying axiom (M5)
      $$\frac{1}{\frac{1}{x}} = \frac{x\frac{1}{x}}{\frac{1}{x}} = x$$
  \end{enumerate}
\end{proof}
\end{ex}

\begin{ex}[4]
Let $\mathnormal{E}$ be a nonempty subset of an ordered set; suppose $\alpha$ is a lower bound of $\mathnormal{E}$ and $\beta$ is an upper bound of $\mathnormal{E}$. Prove that $\alpha \leq \beta$.

\begin{proof}
If $\alpha$ is a lower bound of $\mathnormal{E}$, then for every $x \in \mathnormal{E}$, $\alpha \leq x$. Similarly, if $\beta$ is an upper bound of $\mathnormal{E}$, then for every $x \in \mathnormal{E}$, $\beta \geq x$. Lets assume by contradiction that $\alpha > \beta$. Then $\beta < \alpha$ and $\alpha \leq x$ for every $x \in \mathnormal{E}$, then $\beta \leq x$ for every $x \in \mathnormal{E}$. However, this is a contradiction, since $\beta$ was already defined as $\beta \geq x$ for every $x \in \mathnormal{E}$, which can only be true if $\beta = \alpha$. Therefore, $\alpha \leq \beta$.
\end{proof}
\end{ex}

\begin{ex}[5]
Let $\mathnormal{A}$ be a nonempty set of real numbers which is bounded below. Let -A be the set of all numbers -x, where $x \in \mathnormal{A}$. Prove that $$\inf \mathnormal{A} = -\sup(-\mathnormal{A})$$
\begin{proof}
Take some arbitrary $x \in -\mathnormal{A}$. This means that $-x \in \mathnormal{A}$. Now, the infimum of $\mathnormal{A}$ is defined as some $\alpha \leq y$ for every $y \in \mathnormal{A}$. This means that $$\alpha \leq -x$$ and thus $$-\alpha > x$$ Since $x$ was chosen arbitrarily, this means that $-\alpha$ is greater than $x$ for every $x \in -\mathnormal{A}$. By definition, $-\alpha$ is thus the supremum of $\mathnormal{A}$. Since $\alpha$ was defined as the infimum of $\mathnormal{A}$, $$- \inf\mathnormal{A} = \sup(-\mathnormal{A})$$ which is equivalent to saying $$\inf \mathnormal{A} = -\sup(-\mathnormal{A})$$
\end{proof}
\end{ex}

\begin{ex}[6]
Fix $b > 1$.
    \begin{enumerate}
        \item[(a)] If $m$, $n$, $p$, $q$ are integers, $n>0$, $q>0$, and $r=m/n=p/q$, prove that $$(b^m)^{1/n}=(b^p)^{1/q}.$$
        \item[(b)] Prove that $b^{r+s}=b^rb^s$ if $r$ and $s$ are rational.
        \item[(c)] If $x$ is real, define $B(x)$ to be the set of all numbers $b^t$, where $t$ is rational and $t\le x$. Prove that $$b^r=\sup B(r)$$ when $r$ is rational.
Hence it makes sense to define $$b^x=\sup B(x)$$ for every real $x$.
        \item[(d)] Prove that $b^{x+y}=b^xb^y$ for all real $x$ and $y$.
    \end{enumerate}
    \begin{proof} Parts:
    \begin{enumerate}
        \item[(a)] Since $m/n = p/q$, $mq = pn$, and $1/n = p/mq$. Using substitution, $(b^m)^{1/n} = (b^m)^{p/mq} = (b^{p/q}) = (b^p)^{1/q}$
        \item[(b)] Since $r$ and $s$ are rationals, $m/n = r$ and $p/q = s$, so $$b^{r + s} = b^{m/n + p/q} = b^{\frac{mq + pn}{nq}} = (b^{mq}b^{pn})^{1/nq} = b^{m/n}b^{p/q} = b^rb^s$$
        \item[(c)] First, $b^r$ must be an upper bound for the set $B(r)$, since $r \geq t$ for every $t \leq r$, and thus $b^r \geq b^t$. Now all that is left is to show that $b^r$ is the least upper bound of $B(r)$. Assume by contradiction that it is not. This would mean that there is some $t$ such that $b^t > b^r$, but $t \geq x$ for every x in $B(r)$. However, this is impossible, since $b^r$ is in $B(r)$ as well, and $r \geq t$ and $r \geq x$ for every x in $B(r)$. This is a contradiction, and thus $b^r$ is the supremum of the set.
        \item[(d)]
    \end{enumerate}
    \end{proof}
\end{ex}

\begin{ex}[7]
 Fix $b>1$, $y>0$, and prove that there is a unique real $x$ such that $b^x=y$, by completing the following outline.
    \begin{enumerate}
        \item[(a)] For any positive integer $n$, $b^n-1\ge n(b-1)$.
        \item[(b)] Hence $b-1\ge n(b^{1/n}-1)$.
        \item[(c)] If $t>1$ and $n>(b-1)/(t-1)$, then $b^{1/n}<t$.
        \item[(d)] If $w$ is such that $b^w<y$, then $b^{w+(1/n)}<y$ for sufficiently large $n$; to see this, apply part (c) with $t=y\cdot b^{-w}$.
        \item[(e)] If $w$ is such that $b^w<y$, then $b^{w-(1/n)}>y$ for sufficiently large $n$.
        \item[(f)] Let $A$ be the set of all $w$ such that $b^w<y$, and show that $x=\sup A$ satisfies $b^x=y$.
        \item[(g)] Prove that this $x$ is unique.
    \end{enumerate}
\begin{proof}
\begin{enumerate}
    \item[(a)] \textbf{TODO}
\end{enumerate}
\end{proof}
\end{ex}

\begin{ex}[8]
Prove that no order can be defined in the complex field that turns it into an ordered field.
\begin{proof}
For the complex field to be an ordered field, it must have the property that $x^2 > 0$ if $x \in \mathbb{C}$ and $x \neq 0$. However, $(0,1) \neq 0$, but $(0,1) * (0,1) = (-1, 0) = -1$. This violates the above property, and thus the complex field cannot be ordered.
\end{proof}
\end{ex}

\begin{ex}[9]
Suppose $z = a + bi$, $w = c + di$. Define $z < w$ if $a < c$ or $a = c$ and $b < d$. Prove that this is an order. Does it have the least upper bound property?
\begin{proof}
To prove that this is an order, it must satisfy two conditions. First, $z < w$, $z > w$, and $z = w$ must all be distinct. Now, since the set of reals is ordered, if $a < c$, $z < w$, and if $a > c$, $z > w$. If $a = c$, then the check falls to $b$ and $d$. The same logic follows, and thus $z < w$, $z > w$, and $z = w$ are all distinct. Second, if $x < y$ and $y < z$, then $x < z$. Using a new $x = e + fi$, if $a < c$ and $c < e$, then \textbf{TODO}
\end{proof}
\end{ex}

\begin{ex}
Exercise 10: Suppose $z=a+ib$, $w=u+iv$, and $$a=\biggl(\frac{|w|+u}{2}\biggr)^{1/2},\quad\quad b=\biggl(\frac{|w|-u}{2}\biggr)^{1/2}.$$ Prove that $z^2=w$ if $v\ge 0$ and that $(\overline{z})^2=w$
\end{ex}
\begin{proof}
\textbf{TODO}
\end{proof}
\end{document}