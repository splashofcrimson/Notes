\documentclass[a4paper]{article}

\def\nterm {Fall}
\def\nyear {2019}
\def\nlecturer {W. Rudin}
\def\ncourse {Honors Analysis I}

\RequirePackage{etex}
\makeatletter
\ifx \nauthor\undefined
  \def\nauthor{Eashan Garg}
\else
\fi

\author{Based on Principles of Mathematical Analysis \nlecturer \\\small Notes/Solutions by \nauthor}
\date{\nterm\ \nyear}

\usepackage{alltt}
\usepackage{amsfonts}
\usepackage{amsmath}
\usepackage{amssymb}
\usepackage{amsthm}
\usepackage{booktabs}
\usepackage{caption}
\usepackage{enumitem}
\usepackage{fancyhdr}
\usepackage{graphicx}
\usepackage{mathdots}
\usepackage{mathtools}
\usepackage{microtype}
\usepackage{multirow}
\usepackage{pdflscape}
\usepackage{pgfplots}
\usepackage{siunitx}
\usepackage{textcomp}
\usepackage{slashed}
\usepackage{tabularx}
\usepackage{tikz}
\usepackage{tkz-euclide}
\usepackage[normalem]{ulem}
\usepackage[all]{xy}
\usepackage{imakeidx}

\makeindex[intoc, title=Index]
\indexsetup{othercode={\lhead{\emph{Index}}}}

\ifx \nextra \undefined
  \usepackage[pdftex,
    hidelinks,
    pdfauthor={Dexter Chua},
    pdfsubject={Cambridge Maths Notes: \ncourse},
    pdftitle={\ncourse},
  pdfkeywords={Cambridge Mathematics Maths Math \nterm\ \nyear\ \ncourse}]{hyperref}
  \title{\ncourse}
\else
  \usepackage[pdftex,
    hidelinks,
    pdfauthor={Dexter Chua},
    pdfsubject={Cambridge Maths Notes: \ncourse\ (\nextra)},
    pdftitle={\ncourse\ (\nextra)},
  pdfkeywords={Cambridge Mathematics Maths Math \nterm\ \nyear\ \ncourse\ \nextra}]{hyperref}

  \title{\ncourse \\ {\Large \nextra}}
  \renewcommand\printindex{}
\fi

\pgfplotsset{compat=1.12}

\pagestyle{fancyplain}
\ifx \ncoursehead \undefined
\def\ncoursehead{\ncourse}
\fi

\lhead{\emph{\nouppercase{\leftmark}}}
\ifx \nextra \undefined
  \rhead{
    \ifnum\thepage=1
    \else
      \ncoursehead
    \fi}
\else
  \rhead{
    \ifnum\thepage=1
    \else
      \ncoursehead \ (\nextra)
    \fi}
\fi
\usetikzlibrary{arrows.meta}
\usetikzlibrary{decorations.markings}
\usetikzlibrary{decorations.pathmorphing}
\usetikzlibrary{positioning}
\usetikzlibrary{fadings}
\usetikzlibrary{intersections}
\usetikzlibrary{cd}

\newcommand*{\Cdot}{{\raisebox{-0.25ex}{\scalebox{1.5}{$\cdot$}}}}
\newcommand {\pd}[2][ ]{
  \ifx #1 { }
    \frac{\partial}{\partial #2}
  \else
    \frac{\partial^{#1}}{\partial #2^{#1}}
  \fi
}
\ifx \nhtml \undefined
\else
  \renewcommand\printindex{}
  \DisableLigatures[f]{family = *}
  \let\Contentsline\contentsline
  \renewcommand\contentsline[3]{\Contentsline{#1}{#2}{}}
  \renewcommand{\@dotsep}{10000}
  \newlength\currentparindent
  \setlength\currentparindent\parindent

  \newcommand\@minipagerestore{\setlength{\parindent}{\currentparindent}}
  \usepackage[active,tightpage,pdftex]{preview}
  \renewcommand{\PreviewBorder}{0.1cm}

  \newenvironment{stretchpage}%
  {\begin{preview}\begin{minipage}{\hsize}}%
    {\end{minipage}\end{preview}}
  \AtBeginDocument{\begin{stretchpage}}
  \AtEndDocument{\end{stretchpage}}

  \newcommand{\@@newpage}{\end{stretchpage}\begin{stretchpage}}

  \let\@real@section\section
  \renewcommand{\section}{\@@newpage\@real@section}
  \let\@real@subsection\subsection
  \renewcommand{\subsection}{\@ifstar{\@real@subsection*}{\@@newpage\@real@subsection}}
\fi
\ifx \ntrim \undefined
\else
  \usepackage{geometry}
  \geometry{
    papersize={379pt, 699pt},
    textwidth=345pt,
    textheight=596pt,
    left=17pt,
    top=54pt,
    right=17pt
  }
\fi

% Theorems
\theoremstyle{definition}
\newtheorem*{aim}{Aim}
\newtheorem*{axiom}{Axiom}
\newtheorem*{claim}{Claim}
\newtheorem*{cor}{Corollary}
\newtheorem*{conjecture}{Conjecture}
\newtheorem*{defi}{Definition}
\newtheorem*{eg}{Example}
\newtheorem*{ex}{Exercise}
\newtheorem*{fact}{Fact}
\newtheorem*{law}{Law}
\newtheorem*{lemma}{Lemma}
\newtheorem*{notation}{Notation}
\newtheorem*{prop}{Proposition}
\newtheorem*{question}{Question}
\newtheorem*{problem}{Problem}
\newtheorem*{rrule}{Rule}
\newtheorem*{thm}{Theorem}
\newtheorem*{assumption}{Assumption}

\newtheorem*{remark}{Remark}
\newtheorem*{warning}{Warning}
\newtheorem*{exercise}{Exercise}

\newtheorem{nthm}{Theorem}[section]
\newtheorem{nlemma}[nthm]{Lemma}
\newtheorem{nprop}[nthm]{Proposition}
\newtheorem{ncor}[nthm]{Corollary}


\renewcommand{\labelitemi}{--}
\renewcommand{\labelitemii}{$\circ$}
\renewcommand{\labelenumi}{(\roman{*})}

\let\stdsection\section
\renewcommand\section{\newpage\stdsection}

% Strike through
\def\st{\bgroup \ULdepth=-.55ex \ULset}


%%%%%%%%%%%%%%%%%%%%%%%%%
%%%%% Maths Symbols %%%%%
%%%%%%%%%%%%%%%%%%%%%%%%%

% Matrix groups
\newcommand{\GL}{\mathrm{GL}}
\newcommand{\Or}{\mathrm{O}}
\newcommand{\PGL}{\mathrm{PGL}}
\newcommand{\PSL}{\mathrm{PSL}}
\newcommand{\PSO}{\mathrm{PSO}}
\newcommand{\PSU}{\mathrm{PSU}}
\newcommand{\SL}{\mathrm{SL}}
\newcommand{\SO}{\mathrm{SO}}
\newcommand{\Spin}{\mathrm{Spin}}
\newcommand{\Sp}{\mathrm{Sp}}
\newcommand{\SU}{\mathrm{SU}}
\newcommand{\U}{\mathrm{U}}
\newcommand{\Mat}{\mathrm{Mat}}

% Matrix algebras
\newcommand{\gl}{\mathfrak{gl}}
\newcommand{\ort}{\mathfrak{o}}
\newcommand{\so}{\mathfrak{so}}
\newcommand{\su}{\mathfrak{su}}
\newcommand{\uu}{\mathfrak{u}}
\renewcommand{\sl}{\mathfrak{sl}}

% Special sets
\newcommand{\C}{\mathbb{C}}
\newcommand{\CP}{\mathbb{CP}}
\newcommand{\GG}{\mathbb{G}}
\newcommand{\N}{\mathbb{N}}
\newcommand{\Q}{\mathbb{Q}}
\newcommand{\R}{\mathbb{R}}
\newcommand{\RP}{\mathbb{RP}}
\newcommand{\T}{\mathbb{T}}
\newcommand{\Z}{\mathbb{Z}}
\renewcommand{\H}{\mathbb{H}}

% Brackets
\newcommand{\abs}[1]{\left\lvert #1\right\rvert}
\newcommand{\bket}[1]{\left\lvert #1\right\rangle}
\newcommand{\brak}[1]{\left\langle #1 \right\rvert}
\newcommand{\braket}[2]{\left\langle #1\middle\vert #2 \right\rangle}
\newcommand{\bra}{\langle}
\newcommand{\ket}{\rangle}
\newcommand{\norm}[1]{\left\lVert #1\right\rVert}
\newcommand{\normalorder}[1]{\mathop{:}\nolimits\!#1\!\mathop{:}\nolimits}
\newcommand{\tv}[1]{|#1|}
\renewcommand{\vec}[1]{\boldsymbol{\mathbf{#1}}}

% not-math
\newcommand{\bolds}[1]{{\bfseries #1}}
\newcommand{\cat}[1]{\mathsf{#1}}
\newcommand{\ph}{\,\cdot\,}
\newcommand{\term}[1]{\emph{#1}\index{#1}}
\newcommand{\phantomeq}{\hphantom{{}={}}}
% Probability
\DeclareMathOperator{\Bernoulli}{Bernoulli}
\DeclareMathOperator{\betaD}{beta}
\DeclareMathOperator{\bias}{bias}
\DeclareMathOperator{\binomial}{binomial}
\DeclareMathOperator{\corr}{corr}
\DeclareMathOperator{\cov}{cov}
\DeclareMathOperator{\gammaD}{gamma}
\DeclareMathOperator{\mse}{mse}
\DeclareMathOperator{\multinomial}{multinomial}
\DeclareMathOperator{\Poisson}{Poisson}
\DeclareMathOperator{\var}{var}
\newcommand{\E}{\mathbb{E}}
\newcommand{\Prob}{\mathbb{P}}

% Algebra
\DeclareMathOperator{\adj}{adj}
\DeclareMathOperator{\Ann}{Ann}
\DeclareMathOperator{\Aut}{Aut}
\DeclareMathOperator{\Char}{char}
\DeclareMathOperator{\disc}{disc}
\DeclareMathOperator{\dom}{dom}
\DeclareMathOperator{\fix}{fix}
\DeclareMathOperator{\Hom}{Hom}
\DeclareMathOperator{\id}{id}
\DeclareMathOperator{\image}{image}
\DeclareMathOperator{\im}{im}
\DeclareMathOperator{\re}{re}
\DeclareMathOperator{\tr}{tr}
\DeclareMathOperator{\Tr}{Tr}
\newcommand{\Bilin}{\mathrm{Bilin}}
\newcommand{\Frob}{\mathrm{Frob}}

% Others
\newcommand\ad{\mathrm{ad}}
\newcommand\Art{\mathrm{Art}}
\newcommand{\B}{\mathcal{B}}
\newcommand{\cU}{\mathcal{U}}
\newcommand{\Der}{\mathrm{Der}}
\newcommand{\D}{\mathrm{D}}
\newcommand{\dR}{\mathrm{dR}}
\newcommand{\exterior}{\mathchoice{{\textstyle\bigwedge}}{{\bigwedge}}{{\textstyle\wedge}}{{\scriptstyle\wedge}}}
\newcommand{\F}{\mathbb{F}}
\newcommand{\G}{\mathcal{G}}
\newcommand{\Gr}{\mathrm{Gr}}
\newcommand{\haut}{\mathrm{ht}}
\newcommand{\Hol}{\mathrm{Hol}}
\newcommand{\hol}{\mathfrak{hol}}
\newcommand{\Id}{\mathrm{Id}}
\newcommand{\lie}[1]{\mathfrak{#1}}
\newcommand{\op}{\mathrm{op}}
\newcommand{\Oc}{\mathcal{O}}
\newcommand{\pr}{\mathrm{pr}}
\newcommand{\Ps}{\mathcal{P}}
\newcommand{\pt}{\mathrm{pt}}
\newcommand{\qeq}{\mathrel{``{=}"}}
\newcommand{\Rs}{\mathcal{R}}
\newcommand{\Vect}{\mathrm{Vect}}
\newcommand{\wsto}{\stackrel{\mathrm{w}^*}{\to}}
\newcommand{\wt}{\mathrm{wt}}
\newcommand{\wto}{\stackrel{\mathrm{w}}{\to}}
\renewcommand{\d}{\mathrm{d}}
\renewcommand{\P}{\mathbb{P}}
%\renewcommand{\F}{\mathcal{F}}


\let\Im\relax
\let\Re\relax

\DeclareMathOperator{\area}{area}
\DeclareMathOperator{\card}{card}
\DeclareMathOperator{\ccl}{ccl}
\DeclareMathOperator{\ch}{ch}
\DeclareMathOperator{\cl}{cl}
\DeclareMathOperator{\cls}{\overline{\mathrm{span}}}
\DeclareMathOperator{\coker}{coker}
\DeclareMathOperator{\conv}{conv}
\DeclareMathOperator{\cosec}{cosec}
\DeclareMathOperator{\cosech}{cosech}
\DeclareMathOperator{\covol}{covol}
\DeclareMathOperator{\diag}{diag}
\DeclareMathOperator{\diam}{diam}
\DeclareMathOperator{\Diff}{Diff}
\DeclareMathOperator{\End}{End}
\DeclareMathOperator{\energy}{energy}
\DeclareMathOperator{\erfc}{erfc}
\DeclareMathOperator{\erf}{erf}
\DeclareMathOperator*{\esssup}{ess\,sup}
\DeclareMathOperator{\ev}{ev}
\DeclareMathOperator{\Ext}{Ext}
\DeclareMathOperator{\fst}{fst}
\DeclareMathOperator{\Fit}{Fit}
\DeclareMathOperator{\Frac}{Frac}
\DeclareMathOperator{\Gal}{Gal}
\DeclareMathOperator{\gr}{gr}
\DeclareMathOperator{\hcf}{hcf}
\DeclareMathOperator{\Im}{Im}
\DeclareMathOperator{\Ind}{Ind}
\DeclareMathOperator{\Int}{Int}
\DeclareMathOperator{\Isom}{Isom}
\DeclareMathOperator{\lcm}{lcm}
\DeclareMathOperator{\length}{length}
\DeclareMathOperator{\Lie}{Lie}
\DeclareMathOperator{\like}{like}
\DeclareMathOperator{\Lk}{Lk}
\DeclareMathOperator{\Maps}{Maps}
\DeclareMathOperator{\orb}{orb}
\DeclareMathOperator{\ord}{ord}
\DeclareMathOperator{\otp}{otp}
\DeclareMathOperator{\poly}{poly}
\DeclareMathOperator{\rank}{rank}
\DeclareMathOperator{\rel}{rel}
\DeclareMathOperator{\Rad}{Rad}
\DeclareMathOperator{\Re}{Re}
\DeclareMathOperator*{\res}{res}
\DeclareMathOperator{\Res}{Res}
\DeclareMathOperator{\Ric}{Ric}
\DeclareMathOperator{\rk}{rk}
\DeclareMathOperator{\Rees}{Rees}
\DeclareMathOperator{\Root}{Root}
\DeclareMathOperator{\sech}{sech}
\DeclareMathOperator{\sgn}{sgn}
\DeclareMathOperator{\snd}{snd}
\DeclareMathOperator{\Spec}{Spec}
\DeclareMathOperator{\spn}{span}
\DeclareMathOperator{\stab}{stab}
\DeclareMathOperator{\St}{St}
\DeclareMathOperator{\supp}{supp}
\DeclareMathOperator{\Syl}{Syl}
\DeclareMathOperator{\Sym}{Sym}
\DeclareMathOperator{\vol}{vol}

\pgfarrowsdeclarecombine{twolatex'}{twolatex'}{latex'}{latex'}{latex'}{latex'}
\tikzset{->/.style = {decoration={markings,
                                  mark=at position 1 with {\arrow[scale=2]{latex'}}},
                      postaction={decorate}}}
\tikzset{<-/.style = {decoration={markings,
                                  mark=at position 0 with {\arrowreversed[scale=2]{latex'}}},
                      postaction={decorate}}}
\tikzset{<->/.style = {decoration={markings,
                                   mark=at position 0 with {\arrowreversed[scale=2]{latex'}},
                                   mark=at position 1 with {\arrow[scale=2]{latex'}}},
                       postaction={decorate}}}
\tikzset{->-/.style = {decoration={markings,
                                   mark=at position #1 with {\arrow[scale=2]{latex'}}},
                       postaction={decorate}}}
\tikzset{-<-/.style = {decoration={markings,
                                   mark=at position #1 with {\arrowreversed[scale=2]{latex'}}},
                       postaction={decorate}}}
\tikzset{->>/.style = {decoration={markings,
                                  mark=at position 1 with {\arrow[scale=2]{latex'}}},
                      postaction={decorate}}}
\tikzset{<<-/.style = {decoration={markings,
                                  mark=at position 0 with {\arrowreversed[scale=2]{twolatex'}}},
                      postaction={decorate}}}
\tikzset{<<->>/.style = {decoration={markings,
                                   mark=at position 0 with {\arrowreversed[scale=2]{twolatex'}},
                                   mark=at position 1 with {\arrow[scale=2]{twolatex'}}},
                       postaction={decorate}}}
\tikzset{->>-/.style = {decoration={markings,
                                   mark=at position #1 with {\arrow[scale=2]{twolatex'}}},
                       postaction={decorate}}}
\tikzset{-<<-/.style = {decoration={markings,
                                   mark=at position #1 with {\arrowreversed[scale=2]{twolatex'}}},
                       postaction={decorate}}}

\tikzset{circ/.style = {fill, circle, inner sep = 0, minimum size = 3}}
\tikzset{scirc/.style = {fill, circle, inner sep = 0, minimum size = 1.5}}
\tikzset{mstate/.style={circle, draw, blue, text=black, minimum width=0.7cm}}

\tikzset{eqpic/.style={baseline={([yshift=-.5ex]current bounding box.center)}}}
\tikzset{commutative diagrams/.cd,cdmap/.style={/tikz/column 1/.append style={anchor=base east},/tikz/column 2/.append style={anchor=base west},row sep=tiny}}

\definecolor{mblue}{rgb}{0.2, 0.3, 0.8}
\definecolor{morange}{rgb}{1, 0.5, 0}
\definecolor{mgreen}{rgb}{0.1, 0.4, 0.2}
\definecolor{mred}{rgb}{0.5, 0, 0}

\def\drawcirculararc(#1,#2)(#3,#4)(#5,#6){%
    \pgfmathsetmacro\cA{(#1*#1+#2*#2-#3*#3-#4*#4)/2}%
    \pgfmathsetmacro\cB{(#1*#1+#2*#2-#5*#5-#6*#6)/2}%
    \pgfmathsetmacro\cy{(\cB*(#1-#3)-\cA*(#1-#5))/%
                        ((#2-#6)*(#1-#3)-(#2-#4)*(#1-#5))}%
    \pgfmathsetmacro\cx{(\cA-\cy*(#2-#4))/(#1-#3)}%
    \pgfmathsetmacro\cr{sqrt((#1-\cx)*(#1-\cx)+(#2-\cy)*(#2-\cy))}%
    \pgfmathsetmacro\cA{atan2(#2-\cy,#1-\cx)}%
    \pgfmathsetmacro\cB{atan2(#6-\cy,#5-\cx)}%
    \pgfmathparse{\cB<\cA}%
    \ifnum\pgfmathresult=1
        \pgfmathsetmacro\cB{\cB+360}%
    \fi
    \draw (#1,#2) arc (\cA:\cB:\cr);%
}
\newcommand\getCoord[3]{\newdimen{#1}\newdimen{#2}\pgfextractx{#1}{\pgfpointanchor{#3}{center}}\pgfextracty{#2}{\pgfpointanchor{#3}{center}}}

\newcommand\qedshift{\vspace{-17pt}}
\newcommand\fakeqed{\pushQED{\qed}\qedhere}

\def\Xint#1{\mathchoice
   {\XXint\displaystyle\textstyle{#1}}%
   {\XXint\textstyle\scriptstyle{#1}}%
   {\XXint\scriptstyle\scriptscriptstyle{#1}}%
   {\XXint\scriptscriptstyle\scriptscriptstyle{#1}}%
   \!\int}
\def\XXint#1#2#3{{\setbox0=\hbox{$#1{#2#3}{\int}$}
     \vcenter{\hbox{$#2#3$}}\kern-.5\wd0}}
\def\ddashint{\Xint=}
\def\dashint{\Xint-}

\newcommand\separator{{\centering\rule{2cm}{0.2pt}\vspace{2pt}\par}}

\newenvironment{own}{\color{gray!70!black}}{}

\newcommand\makecenter[1]{\raisebox{-0.5\height}{#1}}

\mathchardef\mdash="2D

\newenvironment{significant}{\begin{center}\begin{minipage}{0.9\textwidth}\centering\em}{\end{minipage}\end{center}}
\DeclareRobustCommand{\rvdots}{%
  \vbox{
    \baselineskip4\p@\lineskiplimit\z@
    \kern-\p@
    \hbox{.}\hbox{.}\hbox{.}
  }}
\DeclareRobustCommand\tph[3]{{\texorpdfstring{#1}{#2}}}
\makeatother

\begin{document}
\maketitle

\tableofcontents

\section{The real number system}
\subsection{Exercises}
\begin{ex}[1]
If $r$ is a rational $(r \neq 0 )$ and $x$ is irrational, prove that $r + x$ and $rx$ are irrational.

\begin{proof}
If $r$ is a rational, then $r = \frac{p}{q}$, and the claim can be rewritten as $\frac{p}{q} + x$ is irrational. Lets assume by contradiction that this quantity is rational. Then $\frac{p}{q} + x = \frac{a}{b}$. It follows that $$x = \frac{a}{b} - \frac{p}{q} = \frac{aq - bp}{bq}$$ However, this is a contradiction, since $x$ is irrational, which means that $r + x$ must be irrational as well. Using a similar argument, lets assume by contradiction that $\frac{p}{q}x$ is rational. Then $\frac{p}{q}x = \frac{a}{b}$. It follows that $$x = \frac{aq}{bp}$$ However, this is a contradiction, since $x$ is irrational, which means that $rx$ must be irrational as well.
\end{proof}
\end{ex}

\begin{ex}[2]
Prove that there is no rational number whose square is 12.

\begin{proof}
$\sqrt{12}$ is equivalent to $2\sqrt{3}$. By exercise $1$, a rational times an irrational is also irrational. If $\sqrt{3}$ is irrational (since $2 \in \mathbb{N}$, $2 \in \mathbb{Q}$, then it is true that $2\sqrt{3}$. Lets assume by contradiction that $\sqrt{3}$ is rational. Then $$(\frac{p}{q})^2 = 3$$ It can also be assumed that p and q are both relatively prime, since if they are not, an equivalent $\frac{p_1}{q_1}$ exists that's equivalent to $\frac{p}{q}$, but such that $p_1$ and $q_1$ are relatively prime. Now, this can be rewritten as $$p^2 = 3q^2$$ Since $p^2$ is divisible by $3$, and $3$ is prime, $p$ is also divisible by $3$. This means there is some $a$ such that $3a = p$. By some algebra: $$3p^2 = q^2$$ However, this is a contradiction, since both $q$ and $p$ are divisible by 3, although we originally claimed the two were relatively prime. Therefore, there is no rational number whose square is $12$.
\end{proof}
\end{ex}

\begin{ex}[3]
Prove the following proposition given that $x, y, z \in F$, where $F$ is a field.
\begin{enumerate}
    \item[(a)] If $x \neq 0$ and $xy = xz$ then $y = z$
    \item[(b)] If $x \neq 0$ and $xy = x$ then $y = 1$
    \item[(c)] If $x \neq 0$ and $xy = 1$ then $y = \frac{1}{x}$
    \item[(d)] If $x \neq 0$ then $\frac{1}{\frac{1}{x}} = x$
\end{enumerate}
\begin{proof} Parts:
  \begin{enumerate}
      \item[(a)] Applying axiom (M5) $$y = \frac{1}{x}xy = \frac{1}{x}xz = z$$
      \item[(b)] Applying axiom (M5)
      $$y = \frac{1}{x}xy = \frac{1}{x}x = 1$$
      \item[(c)] Applying axiom (M5)
      $$y = \frac{1}{x}xy = \frac{1}{x}(1) = \frac{1}{x}$$
      \item[(d)] Applying axiom (M5)
      $$\frac{1}{\frac{1}{x}} = \frac{x\frac{1}{x}}{\frac{1}{x}} = x$$
  \end{enumerate}
\end{proof}
\end{ex}

\begin{ex}[4]
Let $\mathnormal{E}$ be a nonempty subset of an ordered set; suppose $\alpha$ is a lower bound of $\mathnormal{E}$ and $\beta$ is an upper bound of $\mathnormal{E}$. Prove that $\alpha \leq \beta$.

\begin{proof}
If $\alpha$ is a lower bound of $\mathnormal{E}$, then for every $x \in \mathnormal{E}$, $\alpha \leq x$. Similarly, if $\beta$ is an upper bound of $\mathnormal{E}$, then for every $x \in \mathnormal{E}$, $\beta \geq x$. Lets assume by contradiction that $\alpha > \beta$. Then $\beta < \alpha$ and $\alpha \leq x$ for every $x \in \mathnormal{E}$, then $\beta \leq x$ for every $x \in \mathnormal{E}$. However, this is a contradiction, since $\beta$ was already defined as $\beta \geq x$ for every $x \in \mathnormal{E}$, which can only be true if $\beta = \alpha$. Therefore, $\alpha \leq \beta$.
\end{proof}
\end{ex}

\begin{ex}[5]
Let $\mathnormal{A}$ be a nonempty set of real numbers which is bounded below. Let -A be the set of all numbers -x, where $x \in \mathnormal{A}$. Prove that $$\inf \mathnormal{A} = -\sup(-\mathnormal{A})$$
\begin{proof}
Take some arbitrary $x \in -\mathnormal{A}$. This means that $-x \in \mathnormal{A}$. Now, the infimum of $\mathnormal{A}$ is defined as some $\alpha \leq y$ for every $y \in \mathnormal{A}$. This means that $$\alpha \leq -x$$ and thus $$-\alpha > x$$ Since $x$ was chosen arbitrarily, this means that $-\alpha$ is greater than $x$ for every $x \in -\mathnormal{A}$. By definition, $-\alpha$ is thus the supremum of $\mathnormal{A}$. Since $\alpha$ was defined as the infimum of $\mathnormal{A}$, $$- \inf\mathnormal{A} = \sup(-\mathnormal{A})$$ which is equivalent to saying $$\inf \mathnormal{A} = -\sup(-\mathnormal{A})$$
\end{proof}
\end{ex}

\begin{ex}[6]
Fix $b > 1$.
    \begin{enumerate}
        \item[(a)] If $m$, $n$, $p$, $q$ are integers, $n>0$, $q>0$, and $r=m/n=p/q$, prove that $$(b^m)^{1/n}=(b^p)^{1/q}.$$
        \item[(b)] Prove that $b^{r+s}=b^rb^s$ if $r$ and $s$ are rational.
        \item[(c)] If $x$ is real, define $B(x)$ to be the set of all numbers $b^t$, where $t$ is rational and $t\le x$. Prove that $$b^r=\sup B(r)$$ when $r$ is rational.
Hence it makes sense to define $$b^x=\sup B(x)$$ for every real $x$.
        \item[(d)] Prove that $b^{x+y}=b^xb^y$ for all real $x$ and $y$.
    \end{enumerate}
    \begin{proof} Parts:
    \begin{enumerate}
        \item[(a)] Since $m/n = p/q$, $mq = pn$, and $1/n = p/mq$. Using substitution, $(b^m)^{1/n} = (b^m)^{p/mq} = (b^{p/q}) = (b^p)^{1/q}$
        \item[(b)] Since $r$ and $s$ are rationals, $m/n = r$ and $p/q = s$, so $$b^{r + s} = b^{m/n + p/q} = b^{\frac{mq + pn}{nq}} = (b^{mq}b^{pn})^{1/nq} = b^{m/n}b^{p/q} = b^rb^s$$
        \item[(c)] First, $b^r$ must be an upper bound for the set $B(r)$, since $r \geq t$ for every $t \leq r$, and thus $b^r \geq b^t$. Now all that is left is to show that $b^r$ is the least upper bound of $B(r)$. Assume by contradiction that it is not. This would mean that there is some $t$ such that $b^t > b^r$, but $t \geq x$ for every x in $B(r)$. However, this is impossible, since $b^r$ is in $B(r)$ as well, and $r \geq t$ and $r \geq x$ for every x in $B(r)$. This is a contradiction, and thus $b^r$ is the supremum of the set.
        \item[(d)]
    \end{enumerate}
    \end{proof}
\end{ex}

\begin{ex}[7]
 Fix $b>1$, $y>0$, and prove that there is a unique real $x$ such that $b^x=y$, by completing the following outline.
    \begin{enumerate}
        \item[(a)] For any positive integer $n$, $b^n-1\ge n(b-1)$.
        \item[(b)] Hence $b-1\ge n(b^{1/n}-1)$.
        \item[(c)] If $t>1$ and $n>(b-1)/(t-1)$, then $b^{1/n}<t$.
        \item[(d)] If $w$ is such that $b^w<y$, then $b^{w+(1/n)}<y$ for sufficiently large $n$; to see this, apply part (c) with $t=y\cdot b^{-w}$.
        \item[(e)] If $w$ is such that $b^w<y$, then $b^{w-(1/n)}>y$ for sufficiently large $n$.
        \item[(f)] Let $A$ be the set of all $w$ such that $b^w<y$, and show that $x=\sup A$ satisfies $b^x=y$.
        \item[(g)] Prove that this $x$ is unique.
    \end{enumerate}
\begin{proof}
\begin{enumerate}
    \item[(a)] \textbf{TODO}
\end{enumerate}
\end{proof}
\end{ex}

\begin{ex}[8]
Prove that no order can be defined in the complex field that turns it into an ordered field.
\begin{proof}
For the complex field to be an ordered field, it must have the property that $x^2 > 0$ if $x \in \mathbb{C}$ and $x \neq 0$. However, $(0,1) \neq 0$, but $(0,1) * (0,1) = (-1, 0) = -1$. This violates the above property, and thus the complex field cannot be ordered.
\end{proof}
\end{ex}

\begin{ex}[9]
Suppose $z = a + bi$, $w = c + di$. Define $z < w$ if $a < c$ or $a = c$ and $b < d$. Prove that this is an order. Does it have the least upper bound property?
\begin{proof}
To prove that this is an order, it must satisfy two conditions. First, $z < w$, $z > w$, and $z = w$ must all be distinct. Now, since the set of reals is ordered, if $a < c$, $z < w$, and if $a > c$, $z > w$. If $a = c$, then the check falls to $b$ and $d$. The same logic follows, and thus $z < w$, $z > w$, and $z = w$ are all distinct. Second, if $x < y$ and $y < z$, then $x < z$. Using a new $x = e + fi$, if $a < c$ and $c < e$, then \textbf{TODO}
\end{proof}
\end{ex}

\begin{ex}
Exercise 10: Suppose $z=a+ib$, $w=u+iv$, and $$a=\biggl(\frac{|w|+u}{2}\biggr)^{1/2},\quad\quad b=\biggl(\frac{|w|-u}{2}\biggr)^{1/2}.$$ Prove that $z^2=w$ if $v\ge 0$ and that $(\overline{z})^2=w$
\end{ex}
\begin{proof}
\textbf{TODO}
\end{proof}
\end{document}